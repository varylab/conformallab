% !TEX TS-program = pdflatex
% !TEX encoding = UTF-8 Unicode

% This is a simple template for a LaTeX document using the "article" class.
% See "book", "report", "letter" for other types of document.

\documentclass[11pt]{amsart} % use larger type; default would be 10pt

\usepackage[utf8]{inputenc} % set input encoding (not needed with XeLaTeX)

% For Milnor's Lobachevsy function $\ML$ 
\usepackage[OT2,T1]{fontenc} 
\newcommand{\ML}{\mbox{\fontencoding{OT2}\fontfamily{wncyr}\fontseries{m}\fontshape{n}\selectfont L}} 
\usepackage{color}
\usepackage{contour}
\contournumber{32}
\contourlength{0.1em}

%%% Examples of Article customizations
% These packages are optional, depending whether you want the features they provide.
% See the LaTeX Companion or other references for full information.

%%% PAGE DIMENSIONS
\usepackage{geometry} % to change the page dimensions
\geometry{a4paper} % or letterpaper (US) or a5paper or....
% \geometry{margin=2in} % for example, change the margins to 2 inches all round
% \geometry{landscape} % set up the page for landscape
%   read geometry.pdf for detailed page layout information

\usepackage{graphicx} % support the \includegraphics command and options

% \usepackage[parfill]{parskip} % Activate to begin paragraphs with an empty line rather than an indent

%%% PACKAGES
\usepackage{booktabs} % for much better looking tables
\usepackage{array} % for better arrays (eg matrices) in maths
\usepackage{paralist} % very flexible & customisable lists (eg. enumerate/itemize, etc.)
\usepackage{verbatim} % adds environment for commenting out blocks of text & for better verbatim
\usepackage{subfig} % make it possible to include more than one captioned figure/table in a single float
% These packages are all incorporated in the memoir class to one degree or another...

%%% HEADERS & FOOTERS
\usepackage{fancyhdr} % This should be set AFTER setting up the page geometry
\pagestyle{fancy} % options: empty , plain , fancy
\renewcommand{\headrulewidth}{0pt} % customise the layout...
\lhead{}\chead{}\rhead{}
\lfoot{}\cfoot{\thepage}\rfoot{}


%%% END Article customizations

%%% The "real" document content comes below...

\title{The Direction-Based Flattening Energy}
\author{Stefan Sechelmann}
\date{\today} % Activate to display a given date or no date (if empty),
         % otherwise the current date is printed 

\begin{document}
\maketitle

\noindent For triangle $\mathrm{\it{ijk}}\in T$ define the angle at vertex $i$
\[\beta^i_{\mathrm{\it{jk}}}=\left|\alpha_{\mathrm{\it{ij}}} - \alpha_{\mathrm{\it{ki}}}\right|\]
\begin{figure*}[h!]
\scalebox{0.55}{\input{edge_energy.pdf_t}}
\caption{Energy labels at edge $\mathrm{\it{ij}}$}
\end{figure*}

\noindent The direction-based flattening functional at edge $\mathrm{\it{ij}}\in E$ is defined as
\begin{eqnarray*}
S_{\mathrm{\it{ij}}}(\alpha) &=& \alpha_{\mathrm{\it{ij}}} \sum_{\substack{\mathrm{\it{imn}} \ni i\\ \mathrm{\it{ij}\notin \mathrm{\it{imn}}}}} \left(\log\sin\beta^m_{\mathrm{\it{ni}}} - \log\sin\beta^n_{\mathrm{\it{im}}}\right)
					+ \alpha_{\mathrm{\it{ij}}}\sum_{\substack{\mathrm{\it{jmn}} \ni j\\ \mathrm{\it{ij}\notin \mathrm{\it{jmn}}}}} \left(\log\sin\beta^m_{\mathrm{\it{nj}}} - \log\sin\beta^n_{\mathrm{\it{jm}}}\right)\\
					&& + \ML(\beta^j_{\mathrm{\it{ki}}}) + \ML(\beta^j_{\mathrm{\it{il}}}) + \ML(\beta^i_{\mathrm{\it{lj}}}) + \ML(\beta^i_{\mathrm{\it{jk}}})
\end{eqnarray*}
For boundary vertices drop the corresponding sums and $\ML$ terms.


\noindent The gradient of $S$ is given by
\begin{eqnarray*}
\frac{\partial S}{\partial \alpha_{\mathrm{\it{ij}}}} &=& \sum_{\mathrm{\it{ijk}} \ni i} \left(\log\sin\beta^j_{\mathrm{\it{ki}}} - \log\sin\beta^k_{\mathrm{\it{ij}}}\right)
				+\sum_{\mathrm{\it{jlm}} \ni j} \left(\log\sin\beta^l_{\mathrm{\it{mj}}} - \log\sin\beta^m_{\mathrm{\it{jl}}}\right)
\end{eqnarray*}
For boundary vertices drop the corresponding sum.

\end{document}
